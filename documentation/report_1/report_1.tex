\documentclass{article}
\usepackage{graphicx}
\usepackage{natbib}
\usepackage{amsmath}
\setlength\parindent{12pt}
\usepackage[export]{adjustbox}
\usepackage[utf8]{inputenc}
\usepackage{polski}
\usepackage[polish]{babel}
\usepackage[top=1in, bottom=1.25in, left=1.25in, right=1.25in]{geometry}
\usepackage{listings}
\usepackage{indentfirst}
\usepackage{multicol}
\usepackage{color}
\usepackage{float}

%----------------------------------------------------------------
\title{Programowanie zaawansowanych aplikacji webowych}
\author{Jan \textsc{Pajdak} \\ Wojciech \textsc{Słowiński}}
\date{\today}

\begin{document}
\maketitle
\begin{center}
\begin{tabular}{l r}
Prowadzący: Mgr inż. Marcin Jodłowiec &  \\
Grupa zajęciowa: Z00-82b &
\end{tabular}
\end{center}
\tableofcontents

%----------------------------------------------------------------
%   \begin{multicols}{2}
%   	\begin{itemize}
%   		\item
%   	\end{itemize}
%   \end{multicols}
%
%   \begin{figure}[H]
%   	\includegraphics[width=\textwidth]{./images/samopodobienstwo_4.png}
%   	\centering
%   \end{figure}
%
%   linebreaks:
%   \\ (two backslashes)
%   \newline
%   \hfill \break
%----------------------------------------------------------------
\newpage
\section{Raport 1}
\subsection{Tematyka projektu}
\subsection{Wymagania funkcjonalne}
\subsection{Stos technologiczny}
\begin{table}[]
    \begin{tabular}{|l|l|}
        \hline
        Utrwalanie danych & Microsoft SQL Server  \\ \hline
        Backend           & ASP.NET Core          \\ \hline
        Frontend          & Angular 8             \\ \hline
        VCS               & git                   \\ \hline
        Serwer HTTP       & Kestrel / IIS / Azure \\ \hline
    \end{tabular}
\end{table}
Uzasadnienie:
\begin{itemize}
    \item Doświadczenie z wybranymi technologiami
    \item Popularność - duża ilość dostępnej dokumentacji technicznej
    \item Sprawdzone i stabilne rozwiązania
\end{itemize}

%----------------------------------------------------------------
\end{document}